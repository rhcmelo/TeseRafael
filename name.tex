NCL (\textit{Nested Context Language}) é uma linguagem capaz de expressar objetos de mídia (e.g., Texto, imagem, gráfico, áudio, vídeo, animação) e relacionamentos entre esses objetos. Pertencendo ao paradigma declarativo, NCL tem como principal característica uma semântica cujo conceito básico é a maneira simples de determinar significado de cada sentença e não depender de como será usada, ou seja, não considera os detalhes inerentes da semântica de execução. A linguagem foi baseada no modelo NCM (\textit{Nested Context Model})  que possui a representação de todas as entidades necessárias para a especificação de um documento hipermídia. A maioria destas entidades foram representadas na linguagem NCL tais como nós, links e nós de composição. Nós de composição NCM agrupam conjuntos de nós e links, que podem ser compostos, recursivamente. Os nós representam os conteúdos que serão exibidos além de informações de contexto e os links representam o relacionamentos entre os objetos. Além de estar do paradigma declarativo, a linguagem também é dirigida eventos, ou seja, as ações são iniciadas se determinado evento acontecer. Desta forma, existe um conjunto de eventos pré-estabelecido que podem ser disparados durante a exibição de um documento hipermídia e que podem tratados. Inclusive eventos assíncronos como por exemplo a interação com usuário. 
Porém, podemos notar que tanto no modelo NCM quanto na linguagem NCL somente as interações tradicionais do usuários foram contempladas com seleção por meio do controle remoto ou teclado. Hoje, é possível captar vários tipos de interação como voz, reconhecimento de expressão facial, gesto, etc. e que ainda não foram representadas. 
