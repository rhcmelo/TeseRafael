\chapter{Introdução} \label{cap:cap1}

Nas anestesias raquidianas os anestesistas dependem do seu sentimento tátil durante a inserção da agulha no paciente para a correta identificação do local de aplicação do líquido anestésico. O local de aplicação da raquidiana é conhecidos como espaço subaracnóideo \cite{Miller2009}). Para que o anestesista reconheça a chegada da agulha neste local ele precisa reconhecer os tecidos ultrapassados por ela. As anestesias possuem técnicas específicas para identificação dos seus espaços de aplicação. Para que os médicos dominem a técnica da anestesia raquidiana é estimado que são necessários 44 ± 6 repetições de execução deste tipo de procedimento. A Figura 1 ilustra dois momentos da anestesia raquidiana .... . 

===== FIGURA ====

O ultrassom é uma ferramenta eficiente para auxilio na determinação do espaço onde a agulha se encontra \cite{Helayel2010}, mas o uso deste tipo de equipamento não é uma realidade em muitos centros do Brasil \cite{Hamaji2016}. O uso deste equipamento, portanto não faz parte do treinamento de muitas faculdades de medicina para anestesias raquidianas. 


\section{Ideia Central}


\section{Objetivos}
\label{sec:objetivos}



\section{Contribuições da Tese}
\label{sec:contribuicoes}



\section{Estrutura da Tese}
\label{sec:estrutura}

O restante do texto está estruturado da seguinte forma. O Capítulo~\ref{cap:cap2} comenta os principais conceitos e tecnologias envolvidas no desenvolvimento do ambiente de treinamento proposto.

O Capítulo~\ref{cap:cap3} contém os trabalhos relacionados a esta tese assim como o posicionamento deste trabalho frente aos demais.

No Capítulo~\ref{cap:cap4} é apresentada a proposta de desenvolvimento que foi desenvolvida durantes este trabalho. 

O Capítulo~\ref{cap:cap5} apresenta os experimentos que foram feitos. 

O Capítulo~\ref{cap:cap6} apresenta uma avaliação dos experimentos em relação aos seus resultados.

Por fim, o Capítulo~\ref{cap:cap7} conclui o trabalho, apresentando as conclusões, realçando as contribuições desta tese e apontando os  trabalhos futuros.