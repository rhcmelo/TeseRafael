\chapter{Trabalhos Relacionados} \label{cap:cap3}

Os trabalhos apresentados neste capítulo vão desde abordagens para definição e classificação de multimodalidade  \cite{turunen2009multimodal}, propostas como \cite{guedes2016extending} de uma especificação de interação multimodal e multiusuário, apresentações de padrão como em \cite{Kim:2014aa} com o Padrão MPEG-V, implementações que provem integração de sistemas em \cite{pereira2017middleware} integrando o Ginga-NCL com M-HubM-Hub \cite{talavera2015mobile} até trabalho como \cite{mcgill2015review} que discute e importância e os desafios encontrados para implementação de um TV multi-usuários.  

\section {Multimodalidade}

Normalmente, os sistemas de televisão digital permitem o controle sobre o conteúdo transmitido por meio do Guia Eletrônico de Programação (EPG). O EPG consiste em uma grade, onde as colunas representam os canais de televisão, enquanto as linhas representam os intervalos de tempo. Nesse contexto, Turunen et al. \cite{turunen2009multimodal} descrevem como novas modalidades de interface de usuário podem ser usadas para fornecer vários métodos de entrada e saída para interação com o EPG. As modalidades de interação apresentadas em \cite{turunen2009multimodal} incluem gestos, entrada de fala e toque físico. A interface de gestos é usada junto com o teclado do telefone celular. Nesta modalidade de interação, diferentes orientações do celular mudam o funcionamento do teclado. Por exemplo, as teclas de seta vertical para baixo são usadas para mover a seleção no EPG e as teclas de seta horizontal executam funções de \textit{zoom}.

Em \cite{turunen2009multimodal}, a interação por voz inclui comandos para navegação no aplicativo (por exemplo, "Vá para o guia de programação") e para assistir à mídia ("Vá para o canal de notícias"). Os autores também fornecem uma interface de toque físico por meio de uma placa de controle semelhante a um controle remoto clássico, mas em vez de botões, ela possui etiquetas RFID atrás de ícones que se comunicam por meio de celular. Quando um usuário toca um desses objetos com um telefone celular, o comando é lido da tag e entregue ao sistema. De acordo com Turunen et al. \cite{turunen2009multimodal}, os ícones tornam o sistema mais fácil de usar do que um controle remoto clássico. Quando o usuário toca em qualquer ícone do controle, o comando associado é enviado ao EPG.

Na literatura, há diversas soluções propostas para trazer maior interatividade para o usuário, por meio do \textit{middleware} Ginga. Em especial pela adição de dispositivos que funcionem como uma segunda tela de interação com a TV \cite{nery2008desenvolvimento}. Dentre estas soluções, destaca-se a de Batista et al. \cite{batista2010estendendo}. Os autores propõem um módulo para o \textit{middleware} Ginga que possibilita aplicações NCL utilizarem recursos de outros dispositivos secundários para controlar a exibição de conteúdo, interação e captura de eventos. Além disso, a solução permite criar \textit{links} que são acionados de acordo com a identificação do dispositivo secundário. Um característica dessa solução é a utilização de uma API para gerenciamento e registro das classes de dispositivos que irão se comunicar com o dispositivo pai. Um caso de uso é a utilização de celulares para responder pesquisas, controlar a televisão e visualizar conteúdo em conjunto com a aplicação. É importante notar que nesse trabalho relacionado, não são sugeridas novas modalidades de interação, porém, foram lançadas as bases para o desenvolvimento de extensões ao Ginga.

O trabalho de Pedrosa et al. \cite{pedrosa2010componente} especifica e implementa um componente que  permite o recebimento de eventos multimodais por parte de aplicações em C++ residentes no Ginga. Na arquitetura proposta, módulos de comunicação devem enviar um documento XML contendo informações do evento, que por sua vez será traduzido por um \textit{parser} para um objeto, informando a um gerenciador de eventos para notificar as aplicações que manifestaram o interesse pelo evento. No trabalho de Pedrosa et al. \cite{pedrosa2010componente}, a principal contribuição foi utilizar o subsistema Ginga para dar apoio à notificação de eventos de multimodalidade para aplicações C++. Em contraste, no presente trabalho, é proposta uma arquitetura que permite que o tratador de eventos do Ginga reconheça também eventos provenientes de outras modalidade de interação. Com isso, possibilita que aplicações declarativas em NCL sejam construídas com multimodalidade. Além disso, este trabalho propõe a parametrização destes eventos de multimodalidade, de forma a permitir também a interação multimodal por múltiplos usuários.

O trabalho de Carvalho et al. \cite{carvalho2010estendendo} propõe uma arquitetura de software focada na interação entre o usuário e a televisão por meio de sua voz com comandos vocais simples, com o objetivo de substituir funções do controle remoto. Para permitir tais aplicações, os autores propõem diversos novos elementos na linguagem NCL, que são derivados da linguagem VoiceXML. A abordagem de extensão de NCL depende de modificações extensas no formatador da linguagem, porém, no trabalho não é especificado como o formatador deverá reagir. Além disso, não é especificado como o software de reconhecimento poderá ser acoplado ao Ginga-NCL.

Luque et al.\cite{luque2014integration} propõe a inserção de objetos 3D interativos na tela principal da televisão. Além disso, a proposta apresentada em \cite{luque2014integration} também permite a interação por meio de um dispositivo tátil portátil (ou seja, tela secundária). A segunda tela fornece informações adicionais, como estatísticas e visualizações adicionais apresentadas sob demanda. A proposta deles foi aplicada à TV interativa que é capaz de receber e reproduzir conteúdo proveniente de redes de difusão (como a televisão digital terrestre) e da Internet (ou seja, a rede de banda larga). É importante ressaltar que a interação multimodal proposta em \cite{luque2014integration} é caracterizada pela utilização de mais de um dispositivo de interação (joystick e tablet). Porém, em nossa proposta, a interação multimodal está relacionada ao uso de diferentes modos de interação, como voz, gesto ou reconhecimento do olhar.



Guedes et al. \cite{guedes2016extending} propõem uma estrutura de programação de alto nível para apoiar interfaces de usuário multimodais em aplicações multimídia interativas. O framework integra diferentes tipos de modalidades de entrada e saída. Ou seja, suporta modalidades de entrada geradas pelo usuário, como gestos e reconhecedores de voz, e modalidades de saída, como conteúdo audiovisual tradicional, sintetizadores de fala e atuadores. O trabalho modela tipos de entrada multimodais que suportam novas modalidades de entrada, tais como gestos e reconhecimento de voz, e diferentes modalidades de saída, como os conteúdos audiovisuais tradicionais, sintetizadores de voz e atuadores. Para o reconhecimento de voz, os trechos a serem reconhecidos são definidos por meio de arquivos SRGS (\emph{Speech Recognition Grammar Specification}) \cite{srgs}. O \textit{framework} proposto não foi implementado no Ginga-NCL. Essa proposta é bastante diferente da proposta desta tese, onde representam-se diferentes tipos de interação (voz, gesto, etc) através de  diferentes tipos de eventos NCM. Sendo assim, não é necessário criar novos tipos de nós e âncoras NCM para reconhecer a ocorrência desses eventos, como \textit{RecognitionNode} e \textit{RecognitionAnchor} propostos em  \cite{guedes2016extending}. Além disso, fica transparente para o autor o domínio da gramática utilizada pelos reconhecedores.

Pereira et al. \cite{pereira2017middleware} propõem uma infraestrutura de software que tem por objetivo integrar o Ginga-NCL\cite{ABNT:2011aa} com o M-Hub\cite{talavera2015mobile}, um middleware voltado ao domínio da IoT que permite a descoberta dinâmica, estabelecimento de conexão, acesso e distribuição de dados de/para objetos inteligentes. Uma limitação do trabalho é imposta pela limitação das informações que podem ser trocadas via script Lua pois é a única forma de capturar informações publicadas no \textit{broker}. O trabalho não propõe extensões na linguagem NCL de forma que o autor da aplicação possa tratar interações vidas dos dispositivos inteligentes.

Na proposta de Farias et al. \cite{de2020extensions}, o Ginga foi estendido com suporte ao MQTT, para prover integração entre IoT (\textit{Internet of Things}) \cite{gubbi2013internet} e TV digital (TVD). O trabalho desenvolveu aplicativos para validar a estrutura proposta utilizando comunicação MQTT em uma rede doméstica. Informações coletadas da rede doméstica por sensores puderam ser apresentadas na TV. Assim, o trabalho adiciona novos recursos ao \textit{middleware} Ginga, que permitiu a integração de receptores de TVD em aplicativos de IoT, por uma API desenvolvida em NCLua \cite{sant2008nclua}. Porém esta abordagem mantém uma dependência com scripts NCLua onde há uma limitação na troca de informações com as aplicações NCL.

\begin{comment}

\begin{table}[h]
\centering
{
  % distancia entre a linha e o texto
  \renewcommand\arraystretch{1.25}
  \begin{tabular}{|p{1,5cm}|p{6cm}|p{1,5cm}|p{2cm}|} \hline
   \multicolumn{1}{|c|}{Trabalho} & \multicolumn{1}{|c|}{Descrição} & \multicolumn{1}{c|}{Média (ms)} & \multicolumn{1}{c|}{Confiança} \\\hline
    1 & 1 usuário com 5 propriedades &  0  & 1,09E-09    \\\hline
    2 & 5 usuários com 5 propriedades &  1  & 2,04E-09   \\\hline
    3 & 10 usuários com 5 propriedades &  1  & 7,76E-10  \\\hline
    4 & 5 usuários com 10 propriedades &  1  & 1,06E-09  \\\hline
    5 & 10 usuários com 10 propriedades &  1  & 1,47E-09 \\\hline
   \end{tabular}
\caption{Tabela comparativa dos trabalhos relacionados a multimodalidade}
\label{tab:compMultimodalidade}
}
\end{table}
%
\end{comment}
\section {Multiusuário}

O trabalho de Guedes et al. apresentado em \cite{guedes2017extending} se concentra em questões de especificação de interações multiusuário. Mais precisamente, em como o autor define os requisitos de interação do usuário e usa informações de contexto. Não é abordado como o sistema de multimídia deve reunir a descrição do perfil dos usuários e a recuperação de suas variáveis de contexto de tempo de execução. Além disso, todas as características de todos os usuários são armazenadas em um único nó de propriedades do documento multimídia. Neste nó, é armazenado um vetor para cada propriedade que se deseja manter dos usuários. A proposta desta tese se aproxima da abordagem de \cite{guedes2017extending}, à medida que foi criado também uma entidade para representar a classe de usuários. Porém difere em alguns aspectos, pois foram criadas duas entidades para representar os usuários individualmente, uma para identificá-lo e outra para armazenar suas características, sendo representada por um nó de propriedades.



McGill et al. \cite{mcgill2015review} discutem o uso da TV e multitelas, os problemas que esse uso apresenta com relação ao papel da TV em contextos sociais compartilhados e o impacto potencial que novas tecnologias podem ter sobre como usamos e interagimos com a televisão. O uso compartilhado da TV é problemático, tanto do ponto de vista da interação quanto da incapacidade de usar a TV de forma independente, sem afetar o uso de terceiros. Atualmente, os usuários superam esses problemas por meio de várias telas, mas isso também é problemático do ponto de vista social, com potencial para aumento do isolamento digital e falta de compartilhamento com relação aos presentes. McGill et al. demonstram como esses problemas podem ser resolvidos com design de interação de TV, apresentando maneiras em que o uso multiusuário pode ser facilitado por meio de interfaces de uso compartilhado e multivisão, e examinando como a TV pode permitir maior compartilhamento e, portanto, consciência da atividade do dispositivo propondo dois tipos de interface. A interação mediada e a interação concorrente. Com isso, foi demonstrado que a TV é capaz de fazer substancialmente mais do que atualmente é solicitado; ao contrário do uso existente, pode ser de relevância crescente na era multiusuário e multitelas. Outra questão interessante discutida em \cite{mcgill2015review}, foi as possibilidades de que todos os usuários pudessem estar no controle da interação. Dentre elas: \textit{Subsets} onde diferentes membros do grupo tem o controle de diferentes funções exigindoa assim cooperação; \textit{Hierarchy} onde a interação de um mebro sobrepõe a do outro; \textit{plurality} a decisão da seleção são baseadas na maioria dos votos mas a navegabilidade é concorrente e finalmente \textit{Blocking} permite que membros possam bloquear temporariamente outro membro do cobtrole. Em todas as possiblidades  há necessidade a identificação do usuário e é neste ponto que o trabalho conversa com esta tese a medida que também há a identificação do "dono" da interação . Em \cite{mcgill2015review}, os autores não apresentam um arquitetura para prover interação multiusuário, mesmo porque o objetivo do artigo é mostrar como os comportamentos existentes para compartilhamento de uso podem ser reaproveitados e virtualizados.

