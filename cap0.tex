% --- -----------------------------------------------------------------
% --- Elementos usados na Capa e na Folha de Rosto.
% --- EXPRESSÕES ENTRE <> DEVERÃO SER COMPLETADAS COM A INFORMAÇÃO ESPECÍFICA DO TRABALHO
% --- E OS SÍMBOLOS <> DEVEM SER RETIRADOS 
% --- -----------------------------------------------------------------
\autor{RAFAEL HEITOR CORREIA DE MELO} % deve ser escrito em maiúsculo

\titulo{Uma Proposta de uso de Dispositivo Háptico \\para Treinamento de Anestesia raquidiana}

\instituicao{UNIVERSIDADE FEDERAL FLUMINENSE}

\orientador{Aura Conci, D.Sc.}

\local{NITERÓI}

\data{2021} % ano da defesa


\comentario{Tese de Doutorado apresentada ao Programa de Pós-Graduação em Computação da \mbox{Universidade} Federal Fluminense como requisito parcial para a obtenção do Grau de \mbox{Doutor em Computação}. Área de concentração: \mbox{Ciência da Computação}.} %preencha com a sua área de concentração


% --- -----------------------------------------------------------------
% --- Capa. (Capa externa, aquela com as letrinhas douradas)(Obrigatório)
% --- ----------------------------------------------------------------
\capa

% --- -----------------------------------------------------------------
% --- Folha de rosto. (Obrigatório)
% --- ----------------------------------------------------------------
\folhaderosto



\pagestyle{ruledheader}
\setcounter{page}{1}
\pagenumbering{roman}

% --- -----------------------------------------------------------------
% --- Termo de aprovacao. (Obrigatorio)
% --- ----------------------------------------------------------------
\cleardoublepage
\thispagestyle{empty}

\vspace{-60mm}

\begin{center}
   {\large RAFAEL HEITOR CORREIA DE MELO}\\
   \vspace{7mm}

   Uma Proposta de uso de Dispositivo Háptico \\para Treinamento de Anestesia raquidiana\\
   
  \vspace{10mm}
\end{center}

\noindent
\begin{flushright}
\begin{minipage}[t]{8cm}

Tese de Doutorado apresentada ao Programa de Pós-Graduação em Computação da Universidade Federal Fluminense como requisito parcial para a obtenção do \mbox{Grau} de Doutor em Computação. Área de concentração: \mbox{Sistemas de Computação.} %preencha com a sua área de concentração
\end{minipage}
\end{flushright}
\vspace{1.0 cm}
\noindent
Aprovada em MÊS de 2021. \\
\begin{flushright}
  \parbox{10cm}
  {
  \begin{center}
  BANCA EXAMINADORA \\
  \vspace{6mm}
  \rule{11cm}{.1mm} \\
    Profa. D.Sc. Aura Conci - Orientadora, UFF \\
    \vspace{6mm}
  \rule{11cm}{.1mm} \\
     Prof. ---------, UFF \\
   \vspace{6mm}

  \rule{11cm}{.1mm} \\
    Profa. -----------, UFF\\
    \vspace{4mm}
  \rule{11cm}{.1mm} \\
    Prof. ------------, SIGLAUniv\\
  \vspace{4mm}
  \rule{11cm}{.1mm} \\
   ------------------, SIGLAUniv \\
  %\vspace{6mm}
  \end{center}
  }
\end{flushright}
\begin{center}
 % \vspace{4mm}
  Niterói \\
  %\vspace{6mm}
  2021

\end{center}

% --- -----------------------------------------------------------------
% --- Dedicatoria.(Opcional)
% --- -----------------------------------------------------------------
\cleardoublepage
\thispagestyle{empty}
\vspace*{200mm}

\begin{flushright}
{\em 
%Dedicatória(s): Elemento opcional onde o autor presta homenagem ou dedica seu trabalho (ABNT, 2005).
Dedico este trabalho a minha esposa, Evelyn, que sempre me apoiou na direção das minhas conquistas e ao meu filho, Rafael, que, ao chegar me apresentou uma nova forma de amar. 
}
\end{flushright}
\newpage


% --- -----------------------------------------------------------------
% --- Agradecimentos.(Opcional)
% --- -----------------------------------------------------------------
\pretextualchapter{Agradecimentos}
\hspace{5mm}

Agradeço a Deus por me mostrar sempre os caminhos, mesmo nos momentos em que parece que isso não vai acontecer. 

Aos meus pais Julio e Dayse pela preocupação e apoio. Aos meus irmãos Leonardo e Julia pela amizade e companheirismo essenciais nos momentos difíceis.

Agradeço muito a minha orientadora Aura, que mesmo nos momentos de desânimo conseguiu me trazer, em palavras, motivação para seguir em frente.

Ao amigo André que foi essencial em parte dessa caminhada.

À minha família, agradeço a compreensão pelas minhas ausências e minhas desculpas nos momentos de desânimo.


% --- -----------------------------------------------------------------
% --- Resumo em portugues.(Obrigatorio)
% --- -----------------------------------------------------------------
\begin{resumo}

As anestesias raquidianas são procedimentos cegos que dependem do sentimento do médico no decorrer da inserção da agulha para correta identificação do local de aplicação do líquido anestésico. Em grande parte dos centros de treinamento a primeira experiência tátil do médico em treino tende a ser praticada em pacientes reais. Esta prática, apesar de ser efetuada sob supervisão direta, traz riscos para estes pacientes e possíveis inseguranças aos aprendizes. Técnicas alternativas de uso de \textit{phantoms} e cadáveres no treinamento oferecem uma pequena representatividade em relação às variações de pacientes reais. 
Este trabalho propõe o desenvolvimento de um ambiente virtual para simulação do procedimento que envolve anestesias raquidianas. Propõem-se considerar o procedimento de punção com \textit{feedback} tátil e visual usando técnicas de auto treinamento. As sensações táteis do médico em treinamento são simuladas no protótipo através da integração com dispositivo háptico. A geração e visualização dinâmica de modelos de corpos de pacientes baseados em altura e peso também faz parte do processo. A parte do protótipo que envolve a detecção de diferentes sensações de perfuração de tecidos foi validado por alunos da computação. O acesso a membros da comunidade médica foi comprometido pela realidade da pandemia. Finalmente, esta tese  também apresenta um modelo adaptável de um corpo de gestante que possui modelagem de todas as camadas desde a pele das costas até os ossos da coluna vertebral. As camadas de tecido mais variáveis foram modeladas de forma dinâmica de forma a permitir uma maior variabilidade de cenários de treinamento. Esta modelagem pode ainda ser utilizada em outros procedimentos que envolvam a área e camadas modeladas desenvolvidas. 

{\hspace{-8mm} \bf{Palavras-chave}}: Dispositivo háptico, Treinamento médico, Anestesia raquidiana, Realidade virtual, Ambiente virtual, Paciente virtual Simulação, Retorno tátil.

\end{resumo}

% --- -----------------------------------------------------------------
% --- Resumo em lingua estrangeira.(Obrigatorio)
% --- -----------------------------------------------------------------
\begin{abstract}



{\hspace{-8mm} \bf{Keywords}}: Haptics, Medical training, Spinal anesthesia, Virtual reality, Virtual environment, Virtual patient, Simulation, tactile feedback.

\end{abstract}

% --- -----------------------------------------------------------------
% --- Lista de figuras.(Opcional)
% --- -----------------------------------------------------------------
%\cleardoublepage
\listoffigures


% --- -----------------------------------------------------------------
% --- Lista de tabelas.(Opcional)
% --- -----------------------------------------------------------------
\cleardoublepage
%\label{pag:last_page_introduction}
\listoftables
\cleardoublepage

% --- -----------------------------------------------------------------
% --- Lista de abreviatura.(Opcional)
%Elemento opcional, que consiste na relação alfabética das abreviaturas e siglas utilizadas no texto, seguidas das %palavras ou expressões correspondentes grafadas por extenso. Recomenda-se a elaboração de lista própria para cada %tipo (ABNT, 2005).
% --- ----------------------------------------------------------------
\cleardoublepage
\pretextualchapter{Lista de Abreviaturas e Siglas}
\begin{tabular}{lcl}
DEE & : & Distância da pele até o espaço epidural;\\
IMC & : & Índice de Massa Corpórea;\\
RU & : & Reino Unido;\\
RV & : & Realidade Virtual;\\



\end{tabular}
% --- -----------------------------------------------------------------
% --- Sumario.(Obrigatorio)
% --- -----------------------------------------------------------------
\pagestyle{ruledheader}
\tableofcontents


