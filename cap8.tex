\chapter{Considerações Finais} \label{cap:cap8}

\section{Conclusão}
\label{sec:conclusão}

Este trabalho propôs uma extensão ao \textit{middleware} Ginga-NCL denominade NCL 4.0 para representar a interação multimodal em aplicações de DTV com suporte a múltiplos usuários. Novos tipos de eventos NCL foram propostos para representar diferentes modos de interação do usuário e a possibilidade de identificação do eventos no que se refere a quem disparou a interação. Com a abordagem desta tese, um autor de aplicativo de TV digital pode usar apenas o paradigma declarativo para especificar sua aplicação, sem a necessidade de programação de script.

A implementação da proposta no \textit{middleware} Ginga-NCL foi apresentada e uma avaliação de desempenho foi feita usando os eventos \textit{VoiceRecognition}, \textit{GestureRecognition}. Na implementação, foi utilizado o protocolo MQTT e a API do Google para reconhecer a fala. Desenvolvemos aplicações NCL com eventos multimodais e conduzimos três experimentos para avaliar se a extensão proposta fornece um tempo de resposta de interação com o usuário adequado. O primeiro experimento foi conduzido para comparar o tempo de resposta de uma interação \textit{VoiceRecognition} entre o Ginga-NCL padrão e nossa extensão proposta. O experimento mostrou que nossa extensão tem um tempo de resposta consideravelmente mais rápido. O segundo experimento foi conduzido para avaliar a taxa de captura de eventos em nossa extensão Ginga-NCL proposta. Nossos resultados de desempenho com Ginga-NCL estendido mostram que ele se comporta de forma adequada. O terceiro experimento foi um teste de carga para avaliar como a extensão proposta adiciona um atraso à apresentação da mídia no Ginga-NCL. Este experimento foi conduzido com várias publicações sobre tópicos em um \textit{broker} MQTT reservado para receber publicações de dispositivos de reconhecimento de fala e gestos. Tais resultados indicam um bom desempenho, pois apresentam um atraso de aproximadamente 13\textit{ms}. Para avaliação das funcionalidades multiusuário de NCL 4.0 mais dois experimentos foram realizados. O primeiro mede o tempo gasto para carregar todos os usuários verificando as propriedades de um determinado perfil mostrando em seu pior caso um atraso de 1\textit{ms}. O segunda avalia o tempo gasto na criação de links dinâmicos para contemplar todos os usuários tendo seu pior caso um atraso de 5\textit{ms}.

A proposta de extensão também adiciona elementos para especificação de multiusuários possibilitando relacioná-los a eventos de interação e a variáveis de contexto. Tais variáveis poderá conter informações de usuários e/ou perfis de maneira individualizada por meio dos elementos  \textit{<userProfile>} e \textit{<userAgent>}. As informações são armazenadas em elemento também proposto nesta tese. A nova entidade estende \textit{SettingsNode} e se chama \textit{UserSettingsNode}. Podendo associar-se ao \textit{UserAgent}. As informações pode ser trazidas de um arquivo XML por exemplo, mas a arquitetura proposta permita outras especificações. Foi criado uma aplicação para mostrar a viabilidade de termos vários usuários participando da experiência e ter suas informações sendo armazenadas e tradada de forma individualizada. 

Especificamos os novos módulos para a Linguagem NCL, além de alterar alguns módulos existes. Todas as especificações foram feitas em XML Schema.

\section{Trabalhos Futuros}
\label{sec:trabFuturo}

Além das interações digamos intencionais feitas pelo usuário, podemos por meio da integração de sensores às aplicações multimídia, captar as reações do usuário que está consumindo o conteúdo assim como informações do ambiente de execução. A partir desta coleta de dados, a aplicação pode se adaptar, reagindo conforme o estado do usuário ou do ambiente. Então como trabalho futuro temos a implementação da parte da arquitetura de sensoriamento do usuário e do ambiente. Outro o trabalho futuro importante é avaliar as propostas de extensão com autores de aplicações multimídia, através de testes de usabilidade da linguagem. Um outro trabalho futuro é o desenvolvimento de ferramentas de autoria para esses novos conceitos, de forma a definir graficamente uma aplicação com interação multimodal com múltiplos usuários. Por fim, sistemas de interação que possibilitem a fusão das interações e a captação paralela serão considerados em trabalhos futuros. Além disso, estender o HTML com esses novos tipos de eventos também é um trabalho futuro. Ou ainda modelagem da relação dos sensores de ambiente com \textit{AmbientSettingsNode}, principalmente em ambientes que possuem mais de um sensor do mesmo tipo, assim como  modelar os critérios utilizados na leitura global deste ambiente.



