\chapter{Proposta de desenvolvimento} \label{cap:cap4}

A tese desenvolvida visa melhorar o treinamento dos médicos anestesistas no procedimento de anestesia Raquidiana. Este procedimento, em alguns centros hospitalares, é treinado com palestras, aprendizado multimídia \cite{Udani2015}, dicas visuais e mentais \cite{Dreifaldt2006}. Em alguns centros no mundo são utilizados \textit{phantoms} para o treinamento.  Porém, a realidade de muitos destes centros, assim como na UFF, é que a primeira experiência tátil dos novos anestesistas é feita diretamente em um paciente. Esta ocorre após ele assistir a execução real de alguns procedimentos sendo efetuados por médicos experientes. Esta primeira experiência tátil, apesar de acontecer sob supervisão de médicos experientes, pode trazer riscos para estes pacientes e possíveis inseguranças aos aprendizes \cite{Elmofty2017}.

=== CONTINUA ===

\section{Desenvolvimento do ambiente de treinamento}
\label{sec:Dese}

=== CONTINUA ===

\subsection{Simulação de pacientes virtuais}
\label{sec:Pacientes}

=== CONTINUA ===


O próximo capítulo apresenta os experimentos realizados com usuários visando fazer uma avaliação da simulação.